% !TeX root = ../main.tex
\chapter{参考文献}
\label{chp:bib}

\section{导入参考文献}

你有多种方式导入参考文献,最常用的一种是直接从百度学术或谷歌学术中获取文献的 BibTeX 信息,就像这样:

\begin{tcolorbox}
\begin{lstlisting}[language=TeX]
@article{blum2013learning,
  title={A learning theory approach to noninteractive database privacy},
  author={Blum, Avrim and Ligett, Katrina and Roth, Aaron},
  journal={Journal of the ACM (JACM)},
  volume={60},
  number={2},
  pages={1--25},
  year={2013},
  publisher={ACM New York, NY, USA}
}
\end{lstlisting}
\end{tcolorbox}

你只需要将其粘贴到模板根目录下的seumasterthesis.bib文件中,就能够在你的论文中引用该文献。

\section{引用参考文献}

在你的正文中,你有两种方法引用参考文献,其中一种是这样的:

\begin{tcolorbox}
\begin{lstlisting}[language=TeX]
\cite{blum2013learning}
\end{lstlisting}
\end{tcolorbox}

\noindent 它用于实现上标样式的文献引用,就像这样\cite{blum2013learning}。一般我们引用参考文献均采用这种方式。而在另一些情况下,你所引用的参考文献需要在文章或段落中充当语言成分,这时你应当这样引用参考文献:

\begin{tcolorbox}
\begin{lstlisting}[language=TeX]
\citen{blum2013learning}
\end{lstlisting}
\end{tcolorbox}

\noindent 比如在文献综述中,你可能需要这样列举文章所做的工作:

文章\citen{blum2013learning}提出了一种基于非交互式数据库隐私的机器学习理论...

% 在此处把所有示例的参考文献格式均进行引用(用于测试各类文献的排版样式)
网页类:\cite{muzi2020ctex,seugs2015rule,seugs2018license};期刊类:\cite{blum2013learning,JSJX202411005};会议类:\cite{HeZhang-6041,srivastava2011operating,tseng2021codedbulk};书籍类:\cite{feng2019androidsecurity,williams2006gaussian};学位论文类:\cite{gibbs1998bayesian,RN10,RN11}。

\section{参考文献样式}

本模板的参考文献渲染样式基于GB/T 7714-2015国家标准,模板的BST文件来自南京大学的\href{https://github.com/Haixing-Hu}{胡海星}同学提供的\href{https://github.com/CTeX-org/gbt7714-bibtex-style}{CTeX-org},在此对他的工作表示感谢。

此外,工程根目录下的附录3是《中华人民共和国关于参考文献著录规则的国家标准 GB/T 7714-2015》的原文,有需要的同学可以参阅。