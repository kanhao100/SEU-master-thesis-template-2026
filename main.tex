% 学硕
\documentclass[algorithmlist,figurelist,tablelist,nomlist,nocolorlinks]{template/seumasterthesis}
% 参数解析:algorithmlist算法列表, figurelist图片列表, tablelist表格列表, nomlist术语列表,engineer专硕, nocolorlinks取消链接着色, 请根据需要选择你的参数。
% 专硕
% \documentclass[algorithmlist,figurelist,tablelist,nomlist,engineer,nocolorlinks]{template/seumasterthesis}

\usepackage{multirow} % 处理跨行表格数据
\usepackage{float}
\usepackage{lipsum}
\usepackage{pifont}  % 引入 pifont 宏包
\usepackage{enumitem}
% Mac系统取消以下两行代码的注释,
% \usepackage{fontspec} % 加载 fontspec 宏包
% \usepackage{xeCJK}    % 加载 xeCJK 宏包以支持中文

% Mac系统如需更改宋体与Windows保持一直,前往https://github.com/Reanon/SEUThesisLatexTemplate/tree/master/font下载Simsun.ttf后打开安装到系统中即可

\renewcommand{\thefootnote}{\ding{\numexpr171+\value{footnote}}}
% 使用Overleaf需要注释掉下一行,在Windows下需要取消注释
% \setCJKmainfont{simsun.ttc}[AutoFakeBold]
\begin{document}

%% ----------------------------------------------------------------------------
%%                                 Meta Data
%% ----------------------------------------------------------------------------
\categorynumber{TN92} %《中国图书资料分类法》分类法
\UDC{621.3}              %《国际十进分类法UDC》的类号
\secretlevel{公开}        % 学位论文密级分为"公开"、"内部"、"秘密"和"机密"四种
\studentid{220000}      % 学号要完整,前面的零不能省略

%% ----------------------------------------------------------------------------
%%                           Thesis Title and Spine
%% ----------------------------------------------------------------------------
\title
    {东南大学 \LaTeX 论文模板使用手册}        % 论文中文标题
    {如何优雅地撰写硕士研究生毕业论文}         % 论文中文副标题,没有可以空着
    {Southeast University \LaTeX ~Thesis Template User Manual}  % 论文英文标题
    {How to Write a Master Thesis in an Elegant Way}            % 论文英文副标题,没有可以空着

\spine
	% 书脊标题与副标题
    {东南大学 \rotatebox{270}{\raisebox{2.5pt}{LaTeX}} 论文模板使用手册} 
    {}                                                               

%% ----------------------------------------------------------------------------
%%                             Author and Advidor
%% ----------------------------------------------------------------------------
\author
    {王东南}                        % 作者中文姓名
    {XX Xxxx}                  % 作者英文姓名,首字母大写,姓名分开,双字用「-」连接

\advisor
    {王东南 教授}                % 导师中文姓名
    {YY Yyyy}        % 导师英文姓名
    {Prof.}                     % 导师职称
    
% \coadvisor                 % 联合培养导师姓名,没有可以不写
%     {王东南 高工}                  % 导师中文姓名
%     {ZZ Zzzz}             % 导师英文姓名
%     {Eng.}                 % 导师职称 (English), 如教授(Prof.)、副教授(A.P.)

%% ----------------------------------------------------------------------------
%%                              Thesis Defence
%% ----------------------------------------------------------------------------
\engthesistype{应用研究}            % 工程硕士论文类型
\degreetype                        % 学位类型
    {电子信息硕士}
    {Master of Electronic Information}
\major{ \begin{minipage}{\linewidth}\centering 通信工程(含宽带\\网络、移动通信等)\end{minipage}}                 % 一级学科名
\submajor{通信与信息系统}             % 二级学科名(学硕) 研究方向(专硕)
\resdirection{超光速量子纠缠通讯}     % 研究方向(学硕用,专硕请留空)
\defenddate{2025年5月18日}          % 答辩日期 \today
\authorizedate{2025年 月 日}        % 授予学位日期,这个档案袋不需要填 
\committeechair{}               % 答辩委员会主席姓名
\reviewer{}{}                % 两位论文评阅人姓名
\department                        % 学院名称
    {信息科学与工程学院}
    {School of Information Science and Engineering}
% \seuthesisthanks                % 资助信息,没有可以不写
%     {本文的部分工作受国家自然基金 No. zxgg666 的支持与帮助,在此表示感谢。}

% 插入空白页策略开关
% print  : 印刷版,全书强制奇数页(会插空白页),因为前面的封面页较多,所以看起来很多空白页,原来的模板默认选项
% digital: 电子版,仅正文强制奇数页(会插空白页),建议的选项,本模板默认选项
% blind1 : 盲审版1,正文不插空白页但跳页(章节首页保持奇数页)
% blind2 : 盲审版2,正文不插空白页也不跳页 (这会导致章节首页的页眉可能出现"东南大学硕士学位论文"而不是章节标题)
\seuDoublePageMode{digital}

%% ----------------------------------------------------------------------------
%%                                  Cover
%% ----------------------------------------------------------------------------
% ⚠️ 可以在编写论文的时候注释掉封面,加快编译速度
\makebigcover  % 生成A3大封面
\makecover     % 生成小封面
	 
%% ----------------------------------------------------------------------------
%%                          Abstract and Contents
%% ----------------------------------------------------------------------------
\input{chapters/abstract}
\cleardoublepage
\seuTocAndListVSpaceOff   % 控制目录/图表算法目录章节间距,注释掉则需要间隙,不注释则与Word模板一致
\seuTableOfContents      % 生成目录(书签锚点在 cls 内通过 \phantomsection 设置,与摘要、致谢等一致)
\listofothers             % 生成图、表等目录,没有可以不写
 
%% ----------------------------------------------------------------------------
%%                                Main Body
%% ----------------------------------------------------------------------------
\mainmatter                    % 开始正文
\input{chapters/chapter1}      % 第一章:
\input{chapters/chapter2}      % 第二章:
\input{chapters/chapter3}      % 第三章:
\input{chapters/chapter4}      % 第四章:
% !TeX root = ../main.tex
\chapter{参考文献}
\label{chp:bib}

\section{导入参考文献}

你有多种方式导入参考文献,最常用的一种是直接从百度学术或谷歌学术中获取文献的 BibTeX 信息,就像这样:

\begin{tcolorbox}
\begin{lstlisting}[language=TeX]
@article{blum2013learning,
  title={A learning theory approach to noninteractive database privacy},
  author={Blum, Avrim and Ligett, Katrina and Roth, Aaron},
  journal={Journal of the ACM (JACM)},
  volume={60},
  number={2},
  pages={1--25},
  year={2013},
  publisher={ACM New York, NY, USA}
}
\end{lstlisting}
\end{tcolorbox}

你只需要将其粘贴到模板根目录下的seumasterthesis.bib文件中,就能够在你的论文中引用该文献。

\section{引用参考文献}

在你的正文中,你有两种方法引用参考文献,其中一种是这样的:

\begin{tcolorbox}
\begin{lstlisting}[language=TeX]
\cite{blum2013learning}
\end{lstlisting}
\end{tcolorbox}

\noindent 它用于实现上标样式的文献引用,就像这样\cite{blum2013learning}。一般我们引用参考文献均采用这种方式。而在另一些情况下,你所引用的参考文献需要在文章或段落中充当语言成分,这时你应当这样引用参考文献:

\begin{tcolorbox}
\begin{lstlisting}[language=TeX]
\citen{blum2013learning}
\end{lstlisting}
\end{tcolorbox}

\noindent 比如在文献综述中,你可能需要这样列举文章所做的工作:

文章\citen{blum2013learning}提出了一种基于非交互式数据库隐私的机器学习理论...

% 在此处把所有示例的参考文献格式均进行引用(用于测试各类文献的排版样式)
网页类:\cite{muzi2020ctex,seugs2015rule,seugs2018license};期刊类:\cite{blum2013learning,JSJX202411005};会议类:\cite{HeZhang-6041,srivastava2011operating,tseng2021codedbulk};书籍类:\cite{feng2019androidsecurity,williams2006gaussian};学位论文类:\cite{gibbs1998bayesian,RN10,RN11}。

\section{参考文献样式}

本模板的参考文献渲染样式基于GB/T 7714-2015国家标准,模板的BST文件来自南京大学的\href{https://github.com/Haixing-Hu}{胡海星}同学提供的\href{https://github.com/CTeX-org/gbt7714-bibtex-style}{CTeX-org},在此对他的工作表示感谢。

此外,工程根目录下的附录3是《中华人民共和国关于参考文献著录规则的国家标准 GB/T 7714-2015》的原文,有需要的同学可以参阅。      % 第五章:
\input{chapters/chapter6}      % 第六章:

%% ----------------------------------------------------------------------------
%%            Acknowledgement, Appendix, Bibliography and Resume
%% ----------------------------------------------------------------------------
\input{chapters/acknowledgement}    % 致谢
\thesisbib{IEEEfull,reference}               % 生成参考文献

%% 下面一句只是用于提示 TexPad 参考文献位置,正式生成时一定要删除
% \bibliography{reference.bib} % 告诉编译器参考文献所在文件

\input{chapters/appendix}           % 附录
\input{chapters/resume}             % 作者简介
\input{chapters/Committees_list}    % 毕业/学位论文答辩委员会名单
\end{document}
